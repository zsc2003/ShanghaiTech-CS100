\section{Smart Pointer}
\subsection{Introduction to Smart Pointer}
\begin{frame}[fragile]{What is Smart Pointer}
  \begin{itemize}
    \item A Smart Pointer is a wrapper class over a pointer intended to make it more easier and safer to manage memory.
    \item There are two type of Smart Pointer used to manage object:
      \begin{itemize}
        \item \textbf{Shared Pointer} (\texttt{std::shared\_ptr})
        \item \textbf{Unique Pointer} (\texttt{std::unique\_ptr})
      \end{itemize}
    \item There is another special type of Smart Pointer:\\
     \textbf{Weak Pointer}(\texttt{std::weak\_ptr}), which is served as an observer and should not be used alone.\pause
    \item Smart Pointers are also template. To use Smart Pointers, include \texttt{memory} header file.
  \end{itemize}
\end{frame}
\begin{frame}[fragile]{Why We Use Smart Pointers}
  \begin{itemize}
    \item Dynamic memory management is very error-prone
    \begin{itemize}
      \item Forgot to \texttt{delete} memory: Memory Leak
      \item Dereferencing dangling pointer/wild pointer.
      \item \texttt{delete} the same memory twice.
    \end{itemize}\pause
  \begin{cpp}
int *p = new int(42); // allocate memory
auto q = p;           // p and q point 
                      // to the same memory
delete p;             // p and q are both
                      // dangling pointers
p = nullptr;          // Ok. But q is still 
                      // dangling pointer
  \end{cpp}
  \end{itemize}\pause
  It would be nice if the memory can be automatically deleted when it is no longer needed.
\end{frame}
\subsection{Shared Pointer}
\begin{frame}{Shared Pointer Introduction}
  \begin{itemize}
    \item The owner of the resource is responsible for freeing the memory that was allocated.
    \item Shared Pointers have the ability of taking ownership of a pointer and share that ownership.
    \item Share Pointers provide limited garbage-collection based on reference counting.
  \end{itemize}
\end{frame}
\begin{frame}[fragile]{Declare a Shared Pointer}
  \begin{cpp}
shared_ptr<string> p1;      // can point to string.
shared_ptr<vector<int>> p2; // can point to vector<int>.
auto p3 = make_shared<double>(3.14);
// use make_shared to create a shared pointer.
auto p4(p3);  
// copy construct. point to p1 and share the ownership.
shared_ptr<int> p5(new int(42)); 
// point to a newly allocated int.
shared_ptr<int> p6 = new int(43);
// Error: converting from raw pointer to shared pointer 
// is explicit.
  \end{cpp}
\pause
The safest way to create a shared pointer is to use \texttt{make\_shared}.
\end{frame}
\begin{frame}[fragile]{Reference Counting}
  \begin{itemize}
    \item When default constructed, the reference count is 0.
    \item When constructed with a raw pointer, or through \texttt{make\_shared}, the reference count is 1.
    \item The reference count is increased by 1 when copied.
    \item When the reference count is decreased to 0, the memory is freed.
  \end{itemize}
\end{frame}
\begin{frame}[fragile]{Reference Counting}
  \begin{cpp}
shared_ptr<double> p1;               // ref-count: 0 
auto p3 = make_shared<double>(3.14); // ref-count: 1
auto p4(p3); // ref-count: 2
p1 = p3;     // ref-count: 3
p3.reset(new double(2.7)); // reset p3 to a new memory.
// p3 have ref-count 1. p1 and p4 have ref-count 2
p4.reset(); // ref-count: 1
p1.reset(); // ref-count: 0. memory is freed.
  \end{cpp} 
\begin{itemize}\pause
  \item The difference between \texttt{reset()} and assignment operator.
  \item \texttt{reset()} can accept an object which can be converted into \texttt{shared\_ptr} (e.g. raw pointer)
  \item assignment operator can only accept \texttt{shared\_ptr}.
\end{itemize}
\end{frame}
\begin{frame}[fragile]{Pass a Deleter}
  \begin{itemize}
    \item Sometimes we use creater creater and deleter function to manage an object.\pause
    \item For example, suppose we are using a network library.

  \end{itemize}
  \begin{cpp}
struct destination; // connect to where.
struct connection;  // the information of the connection.
connection connect(destination *d); // allocate memory
void disconnect(connection c); // free the memory.
void f(destination &d) {
  connection c = connect(&d);
  // if we forget to call disconnect()
  // we will never be able to disconnect c.
}
        \end{cpp}
\end{frame}
\begin{frame}[fragile]{Pass a Deleter}
  \begin{itemize}
    \item We would like our shared pointer to automaticall call \texttt{disconnect()} for us.
    \begin{cpp}
void f(destination &d) {
  connection c = connect(&d);
  shared_ptr<connection> p(c, disconnect);
  // p will call disconnect() when it is destroyed.
  // this is also exception-safe
}
    \end{cpp}
  \end{itemize}
\end{frame}
\begin{frame}[fragile]{Shared Pointer Pitfalls}
  Mixing shared pointers and raw pointers is begging for trouble.
  \begin{cpp}
void process(shared_ptr<int> ptr) {
  do_something(ptr);
} // ptr was destroyed here.
// If we pass a shared_ptr, the ref-count of ptr 
// is at least 2. Life is good.
shared_ptr<int> p(new int(42)); // ref-count 1.
process(p); // in process, the ref-count is 2.
int i = *p; // correct, the ref-count is 1.
  \end{cpp}
\end{frame}
\begin{frame}[fragile]{Shared Pointer Pitfalls}
  Mixing shared pointers and raw pointers is begging for trouble.
  \begin{cpp}
void process(shared_ptr<int> ptr) {
  do_something(ptr);
} // ptr was destroyed here.
// Although we can't pass raw pointer directly, 
// we can create a temporary shared_ptr.
int *x(new int(1024)); // raw pointer.
process(x); // Error: Converting from raw pointer 
            // to shared pointer is explicit.
process(shared_ptr<int>(x)); 
// correct, but the memeory will be freed.
int j = *x; // UB: x is dangling pointer.
  \end{cpp}
\end{frame}
\begin{frame}[fragile]{Shared Pointer Pitfalls}
\begin{itemize}
  \item \texttt{get()} returns the raw pointer underlying the shared pointer. You should be careful with this function.
  \begin{itemize}
    \item Do not \texttt{delete} the pointer return by \texttt{get()}
    \item Do not use \texttt{get()} to initialize or \texttt{reset()} another shared pointer.
    \begin{cpp}
shared_ptr<int> p1(new int(42)); // ref-count 1
int *q = p1.get(); 
{
  shared_ptr<int> p(q); // ref-count 1
} // p is destroyed and memory is freed.
int* i = *q; // UB: q is dangling pointer.
    \end{cpp}\pause
  \end{itemize}
  \item Do not use the same raw pointer to initialize two shared pointers.
\end{itemize}
\end{frame}

\subsection{Unique Pointer}
\begin{frame}{What is a Unique Pointer?}
  \begin{itemize}
    \item Same as Shared Pointers, Unique Pointers have the ability of taking ownership of a pointer. 
    \item However, Unique Pointer do not share its ownership.
    \item Only one unique pointer can point to an object at the same time.
    \item When the unique pointer is destroyed, the object is destroyed.
  \end{itemize}
\end{frame}
\begin{frame}[fragile]{Declare a Unique Pointer}
  Unique Pointer can be direct initialized or by \texttt{make\_unique}(C++14).
  \begin{cpp}
unique_ptr<int> p1(new int(42));  // correct.
unique_ptr<int> p2(p1); 
// Error: unique_ptr do not support copy construction.
unique_ptr<int> p3; // correct. default-constructed.
p3 = p1; // Error: do not support assignment operator.
unique_ptr<double> p3 = make_unique<double>(3.14); 
unique_ptr<int[]> p4 = make_unique<int[]>(10);
// For array types.
  \end{cpp}
\end{frame}
\begin{frame}[fragile]{Pass a Deleter}
Just like shared pointers, we can also pass a deleter to unique pointer.
\begin{cpp}
void f(destination &d) {
  connection c = connect(&d);
  unique_ptr<connection, decltype(disconnect)*> 
    p(&c, disconnect);
}
\end{cpp}
But the type of the function should be determined at compile time.
\end{frame}
\begin{frame}[fragile]{Change the ownership of a Unique Pointer}
  \begin{itemize}
    \item unique pointer have a function called \texttt{release()}, which is used to give up the ownership of the pointer.
    \item \texttt{release()} returns the pointer underlying the unique pointer and set the unique pointer to null.
    \begin{cpp}
unique_ptr<int> p1(new int(42)); 
auto q = p1.release(); 
// correct. but we have to delete q manually.
    \end{cpp}\pause
    \item We can use \texttt{reset()} and \texttt{release()} together to transfer ownership from one unique pointer to another.
    \begin{cpp}
unique_ptr<int> p3(new int(42));
unique_ptr<int> p4;
p4.reset(p3.release());
    \end{cpp}
  \end{itemize}
\end{frame}
\begin{frame}[fragile]{Return a unique pointer from a function}
How can we return a unique pointer from a function if it does not support copy?
\pause
\\We will return to this problem after we have learnt move semantics.
\end{frame}
\subsection{Weak Pointer}
\begin{frame}{What is Weak Pointer}
\begin{itemize}
  \item Different from shared pointers and unique pointers, weak pointers do not have the ability of taking ownership of a pointer.
  \item Weak pointer should point to a object belongs to a shared pointer, but it does not contribute to the ref-count of the shared pointer.
  \item Even if the weak pointer points to the object, the object still can be destroyed.
\end{itemize}
\end{frame}
\begin{frame}[fragile]{Observe the Object}
  \begin{itemize}
    \item Since weak pointer may point to a object that is already destroyed, we can use \texttt{expired()} to check if the object is still alive.
    \item For the same reason, we should use \texttt{lock()} to get a shared pointer to the object. 
    \item Weak pointer is a good way to solve dangling pointer.
  \end{itemize}
\end{frame}

\begin{frame}[fragile]{Cyclic Dependency}
  Suppose we have two class:
  \begin{columns}
    \begin{column}{0.5\linewidth}
      \begin{cpp}
class A {
  shared_ptr<B> p_; 
 public:
  A() {  cout << "A()\n"; }
  ~A() { cout << "~A()\n"; }
  void setShared(
    shared_ptr<B>& p) {
    p_ = p;
  }
};
      \end{cpp}
    \end{column}
    \begin{column}{0.5\linewidth}
      \begin{cpp}
class B {
  shared_ptr<A> p_;
 public:
  B() { cout << "B()\n"; }
  ~B() { cout << "~B()\n"; }
  void setShared(
    shared_ptr<A>& p) {
    p_ = p;
  }
};
      \end{cpp}
    \end{column}
  \end{columns}
\end{frame}

\begin{frame}[fragile]{Cyclic Dependency}
  What is the output of the function:
  \begin{cpp}
void cyclic() {
  shared_ptr<A> a_ptr(new A);
  shared_ptr<B> b_ptr(new B);
  a_ptr->setShared(b_ptr);
  b_ptr->setShared(a_ptr);
}
  \end{cpp}\pause
  \begin{cpp}
// A()
// B()
  \end{cpp}
  The destructor was not called. There was a memory leak.
\end{frame}

\begin{frame}[fragile]{Cyclic Dependency}
  Now we replace the shared pointer with weak pointer.
  \begin{columns}
    \begin{column}{0.5\linewidth}
      \begin{cpp}
class A {
  weak_ptr<B> p_; 
 public:
  A() { cout << "A()\n"; }
  ~A() { cout << "~A()\n"; }
  void setShared(
    weak_ptr<B> p) { 
    p_ = p;
  }
};
      \end{cpp}
    \end{column}
    \begin{column}{0.5\linewidth}
      \begin{cpp}
class B {
  weak_ptr<A> sP1;
 public:
  B() {  cout << "B()\n"; }
  ~B() { cout << "~B()\n"; }
  void setShared(
    weak_ptr<A> p) {
    p_ = p;
  }
};
      \end{cpp}
    \end{column}
  \end{columns}
  The problem solved.
\end{frame}